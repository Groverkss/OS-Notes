\chapter{Introduction}

\section{Virtualisation}

The OS takes a \textbf{physical} resources (such as the processor
, or memory, or a disk) and transforms it into a more general,
powerful, and easy-to-use \textbf{virtual} for of itself. Thus,
we sometimes refer to the operating system as a \textbf{virtual
machine}. This general technique of transforming is called
\textbf{virtualisation}.

\subsection{CPU Virtualisation}

Turning a single CPU into a seemingly infinite number of CPUs and
thus allowing many programs to seemingly run at once is what
we call \textbf{virtualizing the CPU}.\\

\subsection{Memory Virtualisation}

Each process accesses its own \textbf{virtual address spaces}, which
the OS somehow maps onto the physical memory of the machine. Exactly
how this is accomplished is what we study.

\section{Concurrency}

\textbf{Concurrency} is a conceptual term to refer to a host of 
problems that arise, and must bf addressed when working on
many things at once (i.e. concurrently) in the same program.

\section{Presistence}

The software in the operating system that usually manages the disk
is called the \textbf{file system}; it is this
responsible for storing any
files the user creates in a reliable and efficient manner on the 
disks of the system. The file system is the part of OS in charge
of managin persisten data. What techniques are needed to do so? 
What mechanisms and policies are required to do so with a high
probability?
